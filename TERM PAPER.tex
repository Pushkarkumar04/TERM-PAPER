\documentclass{article}
\usepackage[utf8]{inputenc}
\usepackage{graphicx}
\usepackage{pgf}
\usepackage{pgfpages}

\pgfpagesdeclarelayout{boxed}
{
  \edef\pgfpageoptionborder{0pt}
}
{
  \pgfpagesphysicalpageoptions
  {%
    logical pages=1,%
  }
  \pgfpageslogicalpageoptions{1}
  {
    border code=\pgfsetlinewidth{2pt}\pgfstroke,%
    border shrink=\pgfpageoptionborder,%
    resized width=.95\pgfphysicalwidth,%
    resized height=.95\pgfphysicalheight,%
    center=\pgfpoint{.5\pgfphysicalwidth}{.5\pgfphysicalheight}%
  }%
}

\pgfpagesuselayout{boxed}


\title{CAMERA PILL}

\begin{document}
\date{\vspace{-5ex}}
%\date{}  % Toggle commenting to test

\maketitle

\section*{Abstract}
\textbf{The aim of technology is to make products in a large scale for cheaper prices and increased quality. The current technologies have attained a part of it, but the manufacturing technology is at macro level. The future lies in manufacturing product right from the molecular level. Research in this direction started way back in eighties. At that time manufacturing at molecular and atomic level was laughed about. But due to advent of nanotechnology we have realized it to a certain level. One such product manufactured is pill camera , which is used for the treatment of cancer, ulcer and anemia. It has made revolution in the field of medicine. At that time manufacturing at molecular and atomic level was laughed .But due to advent of nanotechnology we have realized it to a certain level. One such product manufactured is pill camera , which is used for the treatment of cancer, ulcer and anemia. It has made revolution in the field of medicine.This tiny capsule can pass through our body, without causing any harm. It takes pictures of our intestine and transmits the same to the receiver of the Computer analysis of our digestive system. This process can help in tracking any kind of disease related to digestive system.
}
\section*{Keywords}
\textbf{nanotechnology ,pill}

\section{GENERAL}
We have made great progress in manufacturing products. Looking back from where we stand now, we started from flint knives and stone tools and reached the stage where we make such tools with more precision than ever. The leap in technology is great but it is not going to stop here. With our present technology we manufacture products by casting, milling, grinding, chipping and the likes. With these technologies we have made more things at a lower cost and greater precision than ever before. In the manufacture of these products we have been arranging atoms in great thundering statistical herds. All of us know manufactured products are made from atoms. The properties of those products depend on how those atoms are arranged. If we rearrange atoms in dirt, water and air we get grass. The next step in manufacturing technology is to manufacture products at molecular level. The technology used to achieve manufacturing at molecular
level is “NANOTECHNOLOGY”. Nanotechnology is the creation of useful materials, devices and system through manipulation of such miniscule matter (nanometer).Nanotechnology deals with objects measured in nanometers. Nanometer can be visualized as billionth of a meter or millionth of a millimeter or it is 1/80000 width of human hair. These technologies we have made more things at a lower cost and greater precision than before.
\\Trillions of assemblers will be needed to develop products in a viable time frame. In order to create enough assemblers to build consumer goods, some Nano machines called explicators will be developed using self-replication process, will be programmed to build more assemblers. Self-replication is a process in which devices whose diameters are of atomic scales, on the order of nanometers, create copies of themselves. For of self-replication to take place in a constructive manner, three conditions must be met.Once swallowed, an electric current flowing through the UW endoscope causes the fiber to bounce back and forth so that its lone electronic eye sees the whole scene.

\subsection{Image processing}
The image processing then combines all this information to create a two-dimensional color picture.

\section{LITERATURE REVIEW}
\subsection{Historical Overview:}
Manipulation of atoms is first talked about by noble laureate Dr.Richard Feynman long ago in 1959 at the annual meeting of the American Physical Society at the California institute of technology -Caltech and at that time it was laughed about. Nothing was pursued init till 80’s. The technology used to achieve It takes pictures of our intestine and transmits the same to the receiver of the Computer analysis of our digestive.

\begin{figure}[h]
    \centering
    \includegraphics[height=4cm,width=4cm]{1.JPG}
    \caption{nickel crystal board}
\end{figure}

\subsubsection{Engines Of Creation:}
Drexel in the year 1981 through his article “The Engines of Creation”. In 1990, IBM researchers showed that it is possible to manipulate single atoms. They positioned 35 Xenon atoms on the surface of nickel crystal, using an atomic force microscopy instrument. These positioned atoms spelled out the letters” IBM”.

\begin{figure}[h]
    \centering
    \includegraphics[height=4cm,width=4cm]{2.JPG}
    \caption{Manufacturing Products Using Nanotechnology}
\end{figure}

There are three steps to achieving nanotechnology-produced goods: Atoms are he building blocks for all matter in our Universe. All the products that are manufactured are made from atoms.
\\The properties of those products depend of how those atoms are arranged .for e.g. If we rearrange the atoms in coal we get diamonds, if we rearrange the atoms in sand and add a pinch of impurities we get computer chips. Scientists must be able to manipulate individual atoms. This means that they will have to develop a technique to grab single atoms and move them to desired positions. In 1990, IBM researchers showed this by positioning 35 xenon atoms on the surface of a nickel

\\Crystal, using an atomic force microscopy instrument. These positioned atoms spelled out the letters "IBM."The next step will be to develop nanoscopic machines, called assemblers, that can beprogrammed to manipulate atoms and molecules at will. It would take thousands of years for a single assembler to produce any kind of material one atom at a time. Trillions of assemblers will be needed to develop products in a viable time frame. In order to create enough assemblers to build consumer goods, some Nano machines called explicators will be developed using self-replication process, will be programmed to build more assemblers. Self-replication is a process in which devices whose diameters are of atomic scales, on the order of nanometers, create copies of themselves. For of self-replication to take place in a constructive manner, three conditions must be met

\subsection{Nanorobot}
The 1st requirement is that each unit be a specialized machine called Nano robot, one of whose functions is to construct at least one copy of itself during its operational life apart from performing its intended task. An e.g. of self-replicating Nano robot is artificial antibody. In addition to reproducing itself, it seeks and destroys disease causing organism

\subsection{Ingredients}
The 2nd requirement is existence of all energy and ingredients necessary to build complete copies of nanorobot in question. Ideally the quantities of each ingredient should be such that they are consumed in the correct proportion., if the process is intended to befinite , then when desired number of nanorobots has been constructed , there should be nounused quantities of any ingredient remaining.

\subsection{Replication Process}

The 3rd requirement is that the environment be controlled so that the Replication process can proceed efficiently and without malfunctions. Excessive turbulence, temperature extremes, intense radiation, or other adverse circumstances might prevent the proper functioning of the nanorobot and cause the process to fail or falter. Once nanorobots are made in sufficient numbers, the process of most of the nanorobots is changed from self-replication to mass manufacturing of products. The nanorobots are connected and controlled by super computer which has the design details of the product to be manufactured. These nanorobots now work in tandem and start placing each molecules of product to b manufactured in the required position. the process of most of the nanorobots is changed from self-replication to mass manufacturing of products.

\section{PILL AMERA APPLICATION}
\subsection{Pill –Sized Camera:}
Imagine a vitamin pill-sized camera that could travel through your body taking pictures, helping diagnose a problem which doctor previously would have found only through surgery. No longer is such technology the stuff of science fiction films.


\begin{figure}[h]
    \centering
    \includegraphics[height=1cm,width=1cm]{3.JPG}
    \caption{Pill Sized Camera}
\end{figure}

\subsection{Conventional Method:}

Currently, standard method of detecting abnormalities in the intestines is through endoscopic examination in which doctors advance a scope down into the small intestine via the mouth. However, these scopes are unable to reach through all of the 20-foot-long small intestine, and thus provide only a partial view of that part of the bowel. With the help of pill camera not only can diagnoses be made for certain conditions routinely missed by other tests, but disorders can be detected at an earlier stage, enabling treatment before complications develop. However, the amount left behind in the body is less than is absorbed by the average person drinking tap water, according to researchers. Scientific advances in areas such as nanotechnology and gene therapy promise to revolutionize the way we discover and develop drugs, as well as how we diagnose and treat disease. The 'camera in a pill' is one recent development that is generating considerable interest.

\begin{figure}[h]
    \centering
    \includegraphics[height=4cm,width=4cm]{4.JPG}
    \caption{conventional camera 3.3 Diagnostic imaging system}
\end{figure}

The device, called the given Diagnostic Imaging System, comes in capsule form and contains a camera,lights, transmitter and batteries. The capsule has a clear end that allows the camera to view the lining of the small intestine. Capsule endoscopy consists of a disposable video camera encapsulated into a pill like form that is swallowed with water. The wireless camera takes thousands of high-quality digital images within the body as it passes through the entire length of the small intestine. The latest pill camera is sized at 26*11 mm and is capable of transmitting 50,000 color images during its traversal through the digestive system of patient.

\\Video chip consists of the IC CMOS image sensor which is used to take pictures of intestine .The lamp is used for proper illumination in the intestine for taking photos. Micro actuator acts as memory to store the software code that is the pH, temp and pressure instructions. The antenna is used to transmit the images to the receiver. For the detection of reliable and correct.
The tiny cameras are swallowed by patients who want less invasive examinations of their digestive track. Until now U.S. DRAM maker Micron Technology Inc. had been the biggest promoter of the camera-in-a-pill concept, with companies such as Israel's Given Imaging charging as much as dollar 450 for its PillCam. MagnaChip is highlighting the low-light sensitivity of the camera, but provided no specification detail. Usually, an LED flash is used to illuminate the area around the capsule.

\begin{figure}[h]
    \centering
    \includegraphics[height=4cm,width=4cm]{5.JPG}
    \caption{future pill camera}
\end{figure}

\subsection{video chip:}
Video chip consists of the IC CMOS image sensor which is used to take pictures of intestine .The lamp is used for proper illumination in the intestine for taking photos. Micro actuator acts as memory to store the software code that is the instructions. The antenna is used to transmit the images to the receiver. For the detection of reliable and correct information, capsule should be able to designed to transmit several biomedical signals, such as pH, temp and pressure.


\begin{figure}[h]
    \centering
    \includegraphics[height=4cm,width=4cm]{6.JPG}
    \caption{components of capsule}
\end{figure}

\section{ENDOSCOPY PROCEDURE}
\subsection{Swallowed Capsule:}
Capsule is swallowed by the patient like a conventional pill.It takes images as it is propelled forward by peristalsis.A wireless recorder, worn on a belt, receives the image transmitted by the pill.A computer workstation processes the data and produces a continuous still images.

\begin{figure}[h]
    \centering
    \includegraphics[height=4cm,width=4cm]{7.JPG}

\end{figure}
Movement Of capsule Through The Digestive System Produces two images per second,approximately 2,600 high quality images.

\begin{figure}[h]
    \centering
    \includegraphics[height=4cm,width=4cm]{8.JPG}

\end{figure}



The proposed telemetry capsule can simultaneously transmit a video signal and receive a control determining the behavior of the capsule. As a result, the total power consumption of the telemetry capsule can be reduced by turning off the camera power during dead time and separately controlling the LEDs for proper illumination in the intestine. Accordingly, proposed telemetry module for bidirectional and multi-channel communication has the potential applications.


\begin{figure}[h]
    \centering
    \includegraphics[height=4cm,width=4cm]{9.JPG}
\caption{conceptional diagram of bidirectional wireless endoscopy system}
\end{figure}

The capsule is capable of transmitting up to eight hours of video before being naturally expelled. No hospitalization is required. The film is downloaded to a computer workstation and processed using a software program called RAPID (reporting and processing of images and data), also developed by Given Imaging. It condenses the film into a 30-minute video. The software also provides an image of the pill as it passes through the small intestine so the physician can match the image to the location of the capsule. Future capsules to be developed using its basic platform. It is not inconceivable that this same technology can be used to pump medication lallow determination of concentration.


\section{RF EMISSION GUIDELINES}
Per FCC filings, the transmitter operates at either 432.13MHz or 433.94MHz, with minimum-shift-keying modulation. MSK has the general benefits of providing constant-envelope modulation, transmitter simplicity and good spectral efficiency. A simple air coil is the radiating antenna element, tucked into the rounded capsule end opposite the camera. Transmit power is held low to manage power consumption, as the receiver antennas are in close proximity with the waist-worn monitor. 
\\432.13MHz or 433.94MHz, with minimum-shift-keying modulation. MSK has the general benefits of providing constant-envelope modulation, transmitter simplicity and good spectral efficiency. A simple air coil is the radiating antenna element, tucked into the rounded capsule end opposite the camera. Transmit power is held low to manage power consumption, as the receiver antennas are in close proximity with the waist-worn monitor.
\\Nevertheless, FCC filings indicate the PillCam stays within emitted RF guidelines only when the pill is inside the body. The minute or so that it takes the pill to go from activated/depackaged form to ingestion is apparently given a waiver as part of the PillCam's regulatory approval.
\\Image capture, switch and transmitter layers are all fabricated on a single rigid-flex PCB. Delayering the board among the three islands of functionality creates flex circuits to interconnect those regions. The assembly is folded up around the batteries, and a pair of gold-plated coil springs distributes power from the imager layer to the lens/illumination layer through holes in the lens barrel
\\The 8hr PillCam lifetime provides up to 57,000 images at a 2fps rate, with the LEDs flashing only during image capture. The combination of low-power CMOS imagers,

\subsection{Pill camera not so hard for patient to swallow:}
As the miniaturisation of cameras continues apace, more and more innovative products are thrown up, such as this pill camera. Basically a lens on a piece of string (isn't that something that Hell's Angels like to do involving string, bacon and laydeez, and goes by the name of Wolfbagging , the technology costs just $300—far less than a $5,000 endoscope. Developed at the University of Washington, the only person who has tried it out so far is research associate professor Eric Siebel.
\\"Never in your life have you ever swallowed anything and it's still sticking out of your mouth, but once you do it, it's easy," he said of the device. It consists of seven fiber optic cables in a capsule about the size of a painkiller, with a 1.4-mm tether that allows the doctor to move the camera around and pull it back up once the exploration is finished.

\\Testing starts at the Seattle Veterans' Administration hospital next year. Once given the thumbs-up, the reusable gadget (disinfect, rinse, repeat, I guess) is expected to be used in the fight against oesophagal cancer. Normal endoscopes are considerably bigger and can only be swallowed after the patient has been sedated (and liberally greased up, probably).

\subsection{Gastroesophageal reflux disease:}
(GERD), is a backflow of acid-containing fluid from the stomach into the esophagus. If it persists, it can develop into a more serious condition known as Barrett’s esophagus. Barrett’s esophagus is a condition in which cells of the lining of the esophagus become pre-malignant and can lead to a potentially fatal form of cancer known as esophageal adenocarcinoma.

\section{DIGESTIVE TRACK}
\subsection{Small Intestine}
The best of hands the entire small intestine is not visualized. The visit to attach the sensor pads and swallow the capsule will take 30 minutes to an hour. You are able to leave the hospital at this time. the digestive track naturally with the aid of the peristaltic activity of the intestinal muscles. The patient comfortably continues with regular activities throughout the examination without feeling sensations resulting from the capsule's passage.
\subsection{Uses:}
 Crohn's Disease.
 malabsorption Disorders.
 Tumors of the small intestine & Vascular Disorders.
 Ulcerative Colitis
 Medication Related To Small Bowel Injury
 
 \subsection{Advantages:}
  Biggest impact on the medical industry
  Nanorobots can perform delicate surgeries.
 They can also change the physical appearance.
 They can slow or reverse the aging process.
 Used to shrink the size of components.


\section{CONCLUSION}
The given endoscopy capsule is a pioneering concept for medical technology of the 21st century.The endoscopy system is the first of its kind to be able to provide non-invasive imaging of the entire small intestine.It has revolutionized the field of diagnostic imaging to a great extent and has proved to be of great help to physicians all over the world.
\\Though nanotechnology has not evolved to its full capacity yet the first rung of products have already made an impact on the market. In the near future most of the conventional manufacturing processes will be replaced with a cheaper and better manufacturing process “nanotechnology”. Scientists predict that this is not all nanotechnology is capable of. They even foresee that in the decades to come, with the help of nanotechnology one can make hearts, lungs, livers and kidneys, just by providing coal, water and some impurities and even prevent the aging effect. Nanotechnology has the power to revolutionize the world of production, but it is sure to increase unemployment.
\\Nanotechnology can be used to make miniature explosives, which would create havoc in human lives. Every new technology that comes opens new doors and horizons but closes some. The same is true with nanotechnology too.
\\You will need to return at the time your nurse gives you. The study takes 8 hours. The capsule most often will pass in your bowel movement.
 \section{REFERENCES}
 
 [1]. Mishra, R. Kayak, S. Verna, K. and Singh, D. (2011) ‘Survey on Techniques to Resolve Problems Associated with RTS/CTS Mechanism’, Proc. Int’l Conf. Comm., Computing and Security (ICCCS), Vol.3, No.1, ISSN 2250-3501.
 
 [2]. R. Mishra, S. Nayak, K. Verma, D. Singh, “Survey on techniques to resolve problems associated with RTS/CTS mechanism”, in Pro .of ICCCS, 2011.
\end{document}